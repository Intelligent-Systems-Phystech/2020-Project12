\documentclass[12pt,twoside]{article}
\usepackage{jmlda}
\title
%    [Обучение машинного перевода без параллельных текстов] % Краткое название; не нужно, если полное название влезает в~колонтитул
{Обучение машинного перевода без параллельных текстов}
\author
%    [Ямалутдинов~А.\,В.] % список авторов для колонтитула; не нужен, если основной список влезает в колонтитул
{Ямалутдинов~А.\,В.,  Стрижов~В.\,В., Бахтеев~О.\,Ю.} % основной список авторов, выводимый в оглавление
\email
{yamalutdinov.av@phystech.ru, strijov@ccas.ru, bakhteev@phystech.edu}
\organization
{$^1$ Московский физико-технический институт, Москва, Россия}%; $^2$Организация}
\abstract
{
	В работе рассматривается метод обучения без учителя для задачи построения системы машинного перевода. 
	Рассматривается подход, основанный на автокодировщиках: каждое предложение переводится кодировщиком в вектор таким образом, чтобы скрытые 
	пространства кодировщиков для разных языков совпадали. 
	Далее, декодировщик переводит полученное векторное представление в предложение на другом языке. 
	Предлагается модификация данного подхода с использованием информации, полученной от систем мультиязычных онтологий: для каждого переводимого предложения 
	строится граф, ребра которого соответствуют отношениям между словами внутри онтологии. Предлагается дополнительный регуляризатора оптимизации —- функция, 
	штрафующая модель перевода за несоответствие графового представления исходного и переведенного предложения.
	Для анализа предложенной модификации проводится эксперимент на языковой паре ''английский-французский''.
    \bigskip
    
	\textbf{Ключевые слова}: \emph {машинный перевод, автокодировщики, нейронные сети}.
}
\begin{document}
	
	\maketitle
	
	\section{Введение}
	В настоящее время одним из основных подходов к задаче машинного перевода является использование глубоких нейронных сетей.
	Традиционный подход к решению данной задачи предполагает, что модель обучается на корпусе параллельных текстов 
	(обучающая выборка состоит из пар предложений на разных языках)\cite{neubig2017translation}. Однако в таких моделях для достижения
	высокого качества перевода необходимы корпуса, состоящие из, как правило, нескольких миллионов параллельных предложений\cite{bahdanu2016corpus}.

	Для некоторых пар языков не существует обучающей выборки достаточного размера. Для решения проблемы машииного перевода между
	данными языками было предложенно несколько подходов, в частности, подход, основанный на автокодировщиках.~\cite{cho2014autoencoders}~\cite{conneau2018unsupervised}
	Рассматриваются автокодировщики, реализованные в виде рекуррентных нейронный сетей, который оптимизируются таким образом, 
	чтобы скрытые пространства автокодировщиков, кодирующих текст на разных языказ, совпадали.

	В данной статье предлагается усовершенствовать предложенный метод, используя данные о мультиязычных онтологиях~\cite{navigli2010wordsense}. 
	Для исходного и переведенного предложения предлагается строить графовые представления, ребра которых соответствуют отношениям между словами.
	Далее эти представления планируется использовать для регуляризации модели, т.е. роста функционала ошибки в случае, когда 
	графовые представления исходного и переведенного предложения значительно отличаются.

	В качестве эксперимента планируется перевод предложений между парой языков ''английский-французский''.

	\section{Постановка задачи}
	
	
	%\begin{State}
	%    Мотивации и~интерпретации наиболее важны для понимания сути работы.
	%\end{State}
	
	%\begin{Theorem}
	%    Не~менее $90\%$ коллег, заинтересовавшихся Вашей статьёй,
	%    прочитают в~ней не~более~$10\%$ текста.
	%\end{Theorem}
	%
	%\begin{Proof}
	%    Причём это будут именно те~разделы, которые не содержат формул.
	%\end{Proof}
	%
	%\begin{Remark}
	%    Выше показано применение окружений
	%    Def, Theorem, State, Remark, Proof.
	%\end{Remark}
	
	
	%\section{Заключение}
	
	%Желательно, чтобы этот раздел был, причём он не~должен дословно повторять аннотацию.
	%Обычно здесь отмечают,
	%каких результатов удалось добиться,
	%какие проблемы остались открытыми.
	
	
	\bibliographystyle{plain}
	\bibliography{references}
	
\end{document}